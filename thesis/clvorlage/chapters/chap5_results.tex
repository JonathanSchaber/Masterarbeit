
\newchap{Results}
\label{chap:5_results}

\section{SRL Evaluation}

\subsection{deISEAR}

\subsubsection{Example 1}

\fig{images/SRLs_deISEAR_1.png}{fig:SRL-PAWS-X-1}{}{10}{}

\subsubsection{Example 2}

\fig{images/SRLs_deISEAR_2.png}{fig:SRL-PAWS-X-1}{}{15}{}

\subsection{PAWS-X}

\begin{landscape}

\subsubsection{Example 1}

\textbf{Sentence 1}

Im Gegenzug [\textsubscript{predicate} gab] Grimoald [\textsubscript{A1} seine Tochter zur
Hochzeit] und gewährte ihm das Herzogtum Spoleto nach dem Tod von Atto.

Im Gegenzug gab Grimoald [\textsubscript{A0} seine Tochter] zur Hochzeit und
[\textsubscript{predicate} gewährte] [\textsubscript{A2} ihm] [\textsubscript{A1} das Herzogtum
Spoleto nach dem Tod von Atto] .

\textbf{Sentence 2}

%[\textsubscript{O} Im Gegenzug] [\textsubscript{predicate} gab] [\textsubscript{O} Grimoald]
[\textsubscript{A1} seine Tochter] [\textsubscript{A3} in die Ehe] [\textsubscript{O} und
gewährte ihm das Herzogtum Spoleto nach dem Tod von Atto] [\textsubscript{O} .]  Im Gegenzug
[\textsubscript{predicate} gab] Grimoald [\textsubscript{A1} seine Tochter] [\textsubscript{A3}
in die Ehe] und gewährte ihm das Herzogtum Spoleto nach dem Tod von Atto.

%[\textsubscript{O} Im Gegenzug gab Grimoald] [\textsubscript{A0} seine Tochter]
[\textsubscript{O} in die Ehe und] [\textsubscript{predicate} gewährte] [\textsubscript{A2}
ihm] [\textsubscript{A1} das Herzogtum Spoleto nach dem Tod von Atto] [\textsubscript{O} .]  Im
Gegenzug gab Grimoald [\textsubscript{A0} seine Tochter] in die Ehe und [\textsubscript{predicate}
gewährte] [\textsubscript{A2} ihm] [\textsubscript{A1} das Herzogtum Spoleto nach dem Tod
von Atto] .


\fig{images/SRLs_PAWS-X_1.png}{fig:SRL-PAWS-X-1}{}{15}{}

\subsubsection{Example 2}

\textbf{Sentence 1}

Camm [\textsubscript{predicate} entschied] , [\textsubscript{A1} dass beide Motoren eingesetzt
werden sollten: Der Tempest Mk 5 hatte den Napier Saber eingebaut, während der Tempest Mk 2
der Bristol Centaurus war] .

Camm entschied, dass [\textsubscript{A1} beide Motoren] [\textsubscript{predicate} eingesetzt]
werden sollten: [\textsubscript{A1} Der Tempest Mk 5 hatte den Napier Saber eingebaut, während]
der Tempest Mk 2 der Bristol Centaurus war.

Camm entschied, dass beide Motoren eingesetzt werden sollten: [\textsubscript{A0} Der Tempest
Mk 5] hatte [\textsubscript{A3} den Napier Saber] [\textsubscript{predicate} eingebaut],
während der Tempest Mk 2 der Bristol Centaurus war.

Camm entschied, dass beide Motoren eingesetzt werden sollten: Der Tempest Mk 5 hatte den
Napier Saber eingebaut, während [\textsubscript{A1} der Tempest Mk 2 der Bristol Centaurus]
[\textsubscript{predicate} war] .

\textbf{Sentence 2}

Camm [\textsubscript{predicate} entschied] , [\textsubscript{A1} dass beide Motoren eingesetz
werden sollten: Der Tempest Mk 5 war mit dem Napier Saber ausgestattet, während der Tempest
Mk 2 den Bristol Centaurus hatte] .

Camm entschied, dass [\textsubscript{A1} beide Motoren] [\textsubscript{predicate} eingesetzt]
werden sollten: [\textsubscript{A1} Der Tempest Mk 5 war mit dem Napier Saber ausgestattet,
während der Tempest Mk 2 den Bristol Centaurus hatte] .

Camm entschied, dass beide Motoren eingesetzt werden sollten: [\textsubscript{A0} Der Tempest
Mk 5] war [\textsubscript{A1} mit dem Napier Saber] [\textsubscript{predicate} ausgestattet]
, während der Tempest Mk 2 den Bristol Centaurus hatte.

Camm entschied, dass beide Motoren eingesetzt werden sollten: Der Tempest Mk 5 war mit dem
Napier Saber ausgestattet, während [\textsubscript{A1} der Tempest Mk 2 den Bristol Centaurus]
[\textsubscript{predicate} hatte] .

\fig{images/SRLs_PAWS-X_2.png}{fig:SRL-PAWS-X-1}{}{15}{}

\end{landscape}

\subsubsection{Example 3}

\fig{images/SRLs_PAWS-X_3.png}{fig:SRL-PAWS-X-1}{}{10}{}

\subsubsection{Example 4}

\fig{images/SRLs_PAWS-X_4.png}{fig:SRL-PAWS-X-1}{}{15}{}

\subsubsection{Example 5}

\fig{images/SRLs_PAWS-X_5.png}{fig:SRL-PAWS-X-1}{}{10}{}

\subsubsection{Example 6}

\fig{images/SRLs_PAWS-X_6.png}{fig:SRL-PAWS-X-1}{}{10}{}

\subsubsection{Example 7}

\fig{images/SRLs_PAWS-X_7.png}{fig:SRL-PAWS-X-1}{}{10}{}

\subsubsection{Example 8}

\fig{images/SRLs_PAWS-X_8.png}{fig:SRL-PAWS-X-1}{}{10}{}

\section{Data Set Results}

\subsection{Testing for Statistical Significance}

``if we rely on empirical evaluation to validate our hypotheses and reveal the correct language processing mechanisms, we better be sure that our results are not coincidental.'' \citep{dror2018hitchhiker}


$\delta(X) = M(A, X) - M(B, X)$

$H_0:\delta(X) \leq 0$

$H_1:\delta(X) > 0$


``It is important to have a method at hand that gives us assurances that the
observed increase in the test score on a test set reflects true improvement in system
quality.'' \citep{koehn2004statistical}

\citet{koehn2004statistical} focus strongly on significance testing in the context of evaluating
on a sub-sample of the test set --- due to expensiveness of testing on the whole set ---
and making statements about the reliability of this subset sample:

``Given a test result of \emph{m} BLEU, we want to compute with a confidence \emph{q} (or
p-level P = 1 - \emph{q}) that the rue BLEU score lies in an interval [\emph{m} - \emph{d},
\emph{m} + \emph{d}].'' \citep{koehn2004statistical}

Since the systems under review here predict on the exact same test set, the assumed independence
of the predictions of the two models holds no longer. \citet{morgan2005statistical} propose
the following algorithm for testing difference significance:

``When the results are better with the new technique, a question arises as to whether
these result differences are due to the new technique actually being better or just due to
chance. Unfortunately, one usually cannot directly answer the question “what is the probability
that the new technique is better given the results on the test data set”'' \citep{yeh2000more}

``But with statistics, one can answer the following proxy question: if the new technique was
actually no different than the old technique (the null hypothesis), what is the probability
that the results on the test set would be at least this skewed in the new technique’s
favor?'' \citep{yeh2000more}

Many evaluation metrics ``have a tendency to underestimate the significance of the results'',
due to their inherent assumption that the compared systems ``produce independent results''
when in reality, they tend to produce ``positively correlated results''. \citep{yeh2000more}


\begin{algorithm}
\caption{Approximate Randomization Algorithm}
\label{alg:approximate-randomization}
	\begin{algorithmic}[1]
    \STATE $O = \{x_1, \dotsc, x_n\} =$ test set
    \STATE $p(M,x) \leftarrow$ prediction from model $M$ on example $x$
    \STATE $O_A = \{p(A,x_1), \dotsc, p(A,x_n)\}$
    \STATE $O_B = \{p(B,x_1), \dotsc, p(B,x_n)\}$
    \STATE $O_{gold} =$ gold labels for $\{x_1, \dotsc, x_n\}$
    \STATE $r \leftarrow 0$
    \STATE $R \leftarrow 0$
    \STATE $rand() \leftarrow$ returns $0$ or $1$, randomly
    \STATE $swap(x,y) \leftarrow$ exchanges elements $x \in A,y \in B$ such that $y \in A, x \in B$
    \STATE $p = 0.05$
    \STATE $e(\hat{Y},Y) =$ evaluation function for gold labels $\hat{Y}$ and predictions $Y$
    \STATE $t_{original} =\ \mid e(O_{gold},O_A) - e(O_{gold},O_B) \mid$
    \WHILE{$R <$ threshold}
      \FORALL{$(a_i, b_i) \in O_A \times O_B \mid i \in I$}
        \IF{$rand() = 0$}
          \STATE $swap(a_i,b_i)$
        \ENDIF
      \ENDFOR
      \STATE $t_{permute} =\ \mid e(O_{gold},O_A') - e(O_{gold},O_B') \mid$
      \IF{$t_{permute} \geq t_{original}$}
        \STATE $r \mathrel{+}= 1$
      \ENDIF
      \STATE $R \mathrel{+}= 1$
    \ENDWHILE
    \IF{$\frac{r+1}{R+1} < p$}
      \STATE system $A$ truly better than system $B$
    \ENDIF
  \end{algorithmic}
\end{algorithm}


\subsubsection{Example Case for XNLI}

Let's consider the case for the merged subtokens setting in the resampled XNLI data set. The test
set contains 1,125 sentence pairs fr which textual entailment must be predicted. From these 1,125
sentence pairs, 398 bear the gold label \emph{contradiction}, 357 are labeled \emph{entailment},
and 370 are \emph{neutral}; so, the class distribution of the set is fairly balanced.

I trained and optimized five algorithms for two architectures on the training and development
set of XNLI: One architecture is the plain ``vanilla'' GRU classifier described in section
XXX, the other is the same GRU architecture enriched with embedded SRLs. The ``vanilla''
system ensemble achieved an accuracy of 66,84\% on the XNLI test set, while the SRL enriched
ensemble scored a 67,64\% --- in other words, the SRL enchriched ensemble performed 0,80\%
better than the ``vanilla'' ensemble.

To check if this difference truly measures the supremacy of the latter model over the first, I
apply the above described algorithm \ref{alg:approximate-randomization} for testing significance
by permuting the actual ensemble predictions. Note that both ensemble models were equally
right or wrong in 1,034 cases out of 1,125.  From this follows, in consequence, that in 92,89\%
of the cases the flipping of predictions between the ensemble models will have no effect.


%%\section{BLEU Scores}
%\label{sec:5_bleuscores}
%
%Table \ref{bleuresults} shows how to use the predefined tab command to have it listed.
%%\tab{#1: label}{#2: long caption}{#3: the table content}{#4: short caption}
%\tab{bleuresults}{BLEU scores of different MT systems}
%{\begin{tabular}{ll|ccc|c}
%language pair		& ABC	& YYY	\\
%\hline
%EN$\rightarrow$DE	& 20.56	& 32.53 \\
%DE$\rightarrow$EN	& 43.35	& 52.53 \\
%\hline
%\end{tabular}
%}{ABC BLEU scores}
%
%And we can reference the large table in the appendix as Table \ref{appendixTable}
%
%\section{Evaluation}
%\label{sec:5_evaluation}
%We saw in section \ref{sec:5_bleuscores} 
%
%We will see in subsection \ref{subsec:5_moreeval} some more evaluations.
%
%\subsection{More evaluation}
%\label{subsec:5_moreeval}
%
%
%\section{Citations}
%Although BLEU scores should be taken with caution (see \citet{Callison-Burch2006})
%or if you prefer to cite like this: \citep{Callison-Burch2006} \ldots
%
%to cite: \cite[30-31]{Koehn2005} \\
%to cite within parentheses/brackets: \citep{Koehn2005}, \citep[30-32]{Koehn2005}\\ %\usepackage[square]{natbib} => square brackets
%
%to cite within the text: \citet{Koehn2005}, \citet[37]{Koehn2005}\\
%only the author(s): \citeauthor{Callison-Burch2006}\\
%only the year: \citeyear{Callison-Burch2006}\\
%
%\section{Graphics}
%
%To include a graphic that appears in the list of figures, use the predefined fig command:\\
%%\fig{#1: filename}{#2: label}{#3: long caption}{#4: width}{#5: short caption}
%\fig{images/Rosetta_Stone.jpg}{fig:rosetta}{The Rosetta Stone}{10}{Rosetta}
%
%%\reffig{#1: label}
%And then reference it as \reffig{fig:rosetta} is easy.
%
%\section{Some Linguistics}
%
%(With the package 'covington')\\
%
%Gloss:
%
%\begin{examples}
% \item \gll The cat sits on the table.
%	    die Katze sitzt auf dem Tisch
%	\glt 'Die Katze sitzt auf dem Tisch.'
%    \glend
%\end{examples}
%
%Gloss with morphology:
%
%\begin{examples}
% \item \gll La gata duerm -e en la cama.
%	    Art.Fem.Sg Katze schlaf -3.Sg in Art.Fem.Sg Bett
%	\glt 'Die Katze schl\"aft im Bett.'
%    \glend
%\end{examples}
%
