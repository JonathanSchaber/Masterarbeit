\newpage
\phantomsection % to get the hyperlinks (and bookmarks in PDF) right for index, list of files, bibliography, etc.
\addcontentsline{toc}{chapter}{Abstract}
\begin{abstract}

\section*{Abstract}

Employing pretrained word embeddings from large language models as input representations has
become state-of-the-art (SOTA) in many Natural Language Processing (NLP) Tasks. Contextualized
representations of modern transformer-based architectures lead to SOTA results on standardized
Natural Language Understanding (NLU) datasets like General Language Understanding Evaluation
(GLUE), often on par with measured human performance. This is all the more astonishing given
that models like Bidirectional Encoder Representations from Transformers (BERT) learn their
embeddings from raw text lacking additional explicit linguistic structures by implementing
self-supervised pre-training. Despite this, BERT embeddings have proven to transfer remarkably
well to NLU tasks through few-shot fine-tuning on small task-specific datasets. Subsequent
research, however, exposed that BERT's NLU capabilities are considerably limited: BERT fails
in certain, often trivial, linguistic contexts to reliably extract the semantic content of a
sentence --- for example, BERT is surprisingly error-prone in recognizing profound changes in
meaning triggered by negative polarity items. In this thesis, I investigate if enriching pure
BERT embeddings with explicit linguistic information counteracts those deficiencies. To this
end, I concatenate pretrained BERT embeddings with numerically encoded, automatically predicted
Semantic Role Labels (SRLs) as input representation for an end-to-end system on downstream
NLU tasks. To assess the increase in semanticity, I devise several head architectures and
compare the performance differences between the enriched and the pure embeddings on the newly
compiled GerGLUE dataset, which comprises various NLU tasks in German. Dataset and SRL quality
are paramount for obtaining conclusive results --- the experiments of this thesis reveald
that translation noise, deficient SRL detection, and insufficient training data can lead to
suboptimal model fitting and selection. Nevertheless, the results indicate that combining
raw, contextualized word embeddings with explicit linguistic information leads to significant
performance increases, suggesting enhanced semantic capabilities of these representations.

% Taking the general
% trend of outsourcing linguistic analysis to machine learning to the extreme, BERT computes
% these embeddings through self-supervised pre-training, completely lacking any linguistic
% framework.
% However, quickly there were flaws and short comings detected, suggesting that BERT fails in certain
% --- often trivial --- contexts reliably recognizing the semantic content of a sentence.
% The results of this thesis indicate that providing BERT with additional linguistic, semanticity
% providing, information leads to a performance improvement on such tasks.
% However, significant gain relies on two key factors: First, the generation of this
% linguistic information is paramount --- in this thesis, this process is automated,
% leading to modest quality of this semantic mark-up, which in turn is reflected in
% noise pruning the generalization capabilities of a model relying on it. Second,
% the suitability of the core data on which the model is trained for specific tasks
% similarly stands and falls with its quality --- e.g. a lot of non-English datasets
% are created by automatically translating English corpora, thereby introducing translation
% artifacts which in turn lead to suboptimal model fitting and selection.




\selectlanguage{ngerman}
\section*{Zusammenfassung}

Und hier sollte die Zusammenfassung auf Deutsch erscheinen.

\selectlanguage{english}
\end{abstract}
\newpage
