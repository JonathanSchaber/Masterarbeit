
\newchap{Semantic Roles}
\label{chap:2_semantic_roles}

\section{Overview}

``The main reason computational systems use semantic roles is to act as a shallow meaning representation that can let us make simple inferences that aren’t possible from the pure surface string of words, or even from the parse tree.'' \cite[p.~375]{jurafsky2019speech}

In the literature, often \cite{gildea2002automatic} is considered to have formally defined the task of automatic SRL.


``Analysis of semantic relations and predicate-argument structure is one of the core pieces of any system for natural language understanding.'' \citep{palmer2010semantic}

% \section{BLEU Scores}
% \label{sec:5_bleuscores}
% 
% Table \ref{bleuresults} shows how to use the predefined tab command to have it listed.
% %\tab{#1: label}{#2: long caption}{#3: the table content}{#4: short caption}
% \tab{bleuresults}{BLEU scores of different MT systems}
% {\begin{tabular}{ll|ccc|c}
% language pair		& ABC	& YYY	\\
% \hline
% EN$\rightarrow$DE	& 20.56	& 32.53 \\
% DE$\rightarrow$EN	& 43.35	& 52.53 \\
% \hline
% \end{tabular}
% }{ABC BLEU scores}
% 
% And we can reference the large table in the appendix as Table \ref{appendixTable}
% 
% \section{Evaluation}
% \label{sec:5_evaluation}
% We saw in section \ref{sec:5_bleuscores} 
% 
% We will see in subsection \ref{subsec:5_moreeval} some more evaluations.
% 
% \subsection{More evaluation}
% \label{subsec:5_moreeval}
% 
% 
% \section{Citations}
% Although BLEU scores should be taken with caution (see \citet{Callison-Burch2006})
% or if you prefer to cite like this: \citep{Callison-Burch2006} \ldots
% 
% to cite: \cite[30-31]{Koehn2005} \\
% to cite within parentheses/brackets: \citep{Koehn2005}, \citep[30-32]{Koehn2005}\\ %\usepackage[square]{natbib} => square brackets
% 
% to cite within the text: \citet{Koehn2005}, \citet[37]{Koehn2005}\\
% only the author(s): \citeauthor{Callison-Burch2006}\\
% only the year: \citeyear{Callison-Burch2006}\\
% 
% \section{Graphics}
% 
% To include a graphic that appears in the list of figures, use the predefined fig command:\\
% %\fig{#1: filename}{#2: label}{#3: long caption}{#4: width}{#5: short caption}
% \fig{images/Rosetta_Stone.jpg}{fig:rosetta}{The Rosetta Stone}{10}{Rosetta}
% 
% %\reffig{#1: label}
% And then reference it as \reffig{fig:rosetta} is easy.
% 
% \section{Some Linguistics}
% 
% (With the package 'covington')\\
% 
% Gloss:
% 
% \begin{examples}
%  \item \gll The cat sits on the table.
% 	    die Katze sitzt auf dem Tisch
% 	\glt 'Die Katze sitzt auf dem Tisch.'
%     \glend
% \end{examples}
% 
% Gloss with morphology:
% 
% \begin{examples}
%  \item \gll La gata duerm -e en la cama.
% 	    Art.Fem.Sg Katze schlaf -3.Sg in Art.Fem.Sg Bett
% 	\glt 'Die Katze schl\"aft im Bett.'
%     \glend
% \end{examples}
% 
