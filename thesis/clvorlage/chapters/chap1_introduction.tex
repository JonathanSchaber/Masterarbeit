\newchap{Introduction}
\label{chap:1_intro}
\section{Motivation}

\epigraph{Verba docent, exempla trahunt.}{\textit{unknown}}

Human language bears some truly mesmerizing features and puzzles, a lot of them are still not
yet understood in all its depths: For example, it is still unclear how children are able to
learn the grammar of their mother tongue from the corrupted and comparatively scarce language
material they are exposed to. %(POS, Chomsky etc) Another astonishing fact about human language
is the overwhelming amount of languages that exist today, even that number was probably much
higher a few centuries ago.  As to how languages evolve, change over time and what trajectories
of possible change may be, lots of questions are still open, and there remains enough work
to do.  But for me, maybe the most trivial and enigmatic trait about human language is that
we actually \emph{understand} each other: That, during a discourse, person X can retrieve
the intentioned meaning of expressions uttered by person Y, and vice versa.  Further, we are
able to logically deduce a whole lot information that is not explicitly stated in a sentence,
and uphold such a state of affairs during the whole conversation.  %Maybe one of the most
enigmatic and fascinating traits of human language is the fact that for us %humans, exchanging
information through language is a meaninglessly working process.  That this is not as trivial
as it might look like on first sight, show the following considerations: Human language is,
when being used, notoriously ambiguous, metaphorical and formally corrupted.

So, every system that claims to process human language in a ... must be able

``It is always precarious to build arguments on inherently vague and general
concepts such as `language', `understanding' and `meaning,' as the resulting
theoretical constructs may become so overly general that they almost become
vacuous.'' \cite{sahlgren2021singleton}

{\color{red} $\rightarrow$ of course, NLU is very strong claim, although to be taken \emph{cum grano salis}:
Be aware of short-comings etc.}

\begin{quote}
  learning a language entails learning the set of symbols and rules that define the
  system. Learning the set of symbols equals vocabulary acqui- sition, while learning
  the rules entails recognizing and formalizing grammati- cal, morphological, and
  syntactic regularities.i We measure these competencies in humans — often indirectly
  — by using various language proficiency tests, such as vocabulary tests, cloze
  tests, reading comprehension, as well as various forms of production, interaction,
  and mediation tests \textelp{}. \cite{sahlgren2021singleton}
\end{quote}

\begin{quote}
  there is an overwhelming body of empirical results to demonstrate that current language models
  have a passable capacity to detect the symbols and rules that define human language. This is not
  what is under dispute in the current debate; what is under dispute is whether such structural
  knowledge about the workings of the language system suffices.
\end{quote}

\subsection{History, Methods, Problems of NLU}


``The `understanding' these models have of language is entirely structural,
and thus does not extend beyond the language system'' \cite{sahlgren2021singleton}

``We can of course always question whether there is any `real' under-
standing going on, but if the absence or presence of this `real' understanding
has no effect on the behavior of the system, it will be a mere epiphenomenon
that need not concern us.'' \cite{sahlgren2021singleton}

``consciousness is not an extra ingredient in addition to the complexity of a system:
consciousness is the complexity of the system. Our point is that understanding is also
not an extra ingredient of a symbol manipulation system: `understanding' is a term we
use to describe the complexity of such a system.'' \cite{sahlgren2021singleton}

``Our point is that when the behavior of an NLU system becomes sufficiently complex, it will be
easier to explain its behavior using intentional terms such as `understanding,' than to use a
purely functional explanation.'' \cite{sahlgren2021singleton}

The subsection of NLP that deals with the semantics, i.e. meanings, of utterances, is NLU.
For quite some time, as in most areas of NLP, systems that addressed NLU problems were
architectures that consisted of carefully hand-written rules that aimed at tackling a specific
problem, such as recognizing textual entailment, coreference resolution, sentiment analysis,
and so on.

From on the 90ies, the so-called emph{statistical revolution} took place, and NLU related
problems were now being addressed by learning patterns from huge data collections.  The main
challenge for engineers and scientists now lay in dicovering suitable features, according to
which the algorithm would hopefully learn helpful patterns for solving the task at hand.

Since now almost a decade, a next stage in NLU and NLP, in general, was entered --- we are
now deep in the neural age of computational linguistics.  In contrast to the statistical
period's main challenge, now the algorithm is even itself learning the features that are the
most informative for a given task.  The human part in the process is to design the overall
model architecture and provide large enough amounts of data that are also of good quality.

In other words,

``The engineering side of computational linguistics, often called natural language processing
(NLP), is largely concerned with building computational tools that do useful things with
language'' \cite{johnson2009statistical}

\subsection{Contextualized Word Embeddings in NLU}

Since the beginning of the neural age, there was the problem as to how could text be numerically
meaningful represented, so that the algorithms can extract meaningful feature patterns and
that there is as little information loss as possible (since a numeric representation is always
an abstraction of the real data, there naturally is some unpreventable information loss).
The solution that was proposed by \cite{mikolov2013distributed} is the approach that is still
in use today in its core idea:

\begin{itemize}
	\item Initialize a random vector for each word in the vocabulary
	\item Train a neural model to learn the best numerical representation of each word
		by giving it a simple task on huge amounts of unlabeled data (like CBOW, next word
		prediction, etc.)
	\item Save those numeric representations and use them in target task at hand
\end{itemize}

While the basic approaches of this approach still hold --- train randomly initialized vectors on
large amounts of unlabeled data with a neural network with a simple training goal ---, some important
changes or additions to today's implementation have been made:

\begin{itemize}
	\item The original word2vec embeddings were \emph{fixed}, in the sense that a word had always
					the same representation, regardless of the context
	\item The neural networks that computed these vectors were quite small (two layers of dimensionality
					300) and could be run on a standard machine.
					Today's models are huge (hundreds of millions of parameters are not unusual) and computationally
					very intensive and cannot be run locally.
	\item Due to the last point, practice has shifted towards pretraining these computationally heavy
					embeddings and finetuning them on the specific task along with it's goal
\end{itemize}

The architecture that has caused the most uproar was probably BERT \cite{devlin2018bert},
an architecture that led to so many variants of it, that it created a whole new field inside
the NLP community --- the BERTology \cite{rogers2020primer}.  These embeddings have also
proved to achieve state-of-the-art results on well-established data sets, such as, e.g.,
GLUE \cite{wang2018glue}.

However, the many studies and experiments that have been carried out exploring the capabilities
and mechanisms behind BERT quickly showed that nevertheless BERT performs on many tasks
surprisingly well, even outperforming all models before it, there are situations, often trivial
looking ones, where BERT desperately fails.

\cite{jin2019bert}, for example, showed that by creating adversarial examples in the test set
--- which were, of course, still valid ---, they could bring down the performance of BERT by
a large margin.

As I laid out before, in the past decades computational linguistics has undergone several
``revolutions'' which, although some people might see this differently, can be described as
moving from a strong emphasis on linguistics to a more data-driven computational discipline.

Furthermore, the introduction of deep learning into computational linguistics has introduced a
so called \emph{black box}; which means essentially that although the underlying formulas and the
architecture of neural nets are well-known --- the mathematics behind them is rather simple ---,
it is nevertheless impossible to determine \emph{what exactly} those models learn from the data.

One way to address the above outlined problems (1) the failure of very sophisticated models
in rather trivial situations and (2) the difficulties of the interpretation of their output
lies in bringing back again the linguistics into the whole picture.

Examples:

\begin{itemize}
	\item Zhang et al SemBERT
	\item Goldberg Syntax to the rescue
	\item ...
\end{itemize}


\section{Research Questions}

The research questions that shall be answered in this thesis, are:
\begin{enumerate}
 \item What do I do?
 \item How do I do it?
 \item And why?
	\item Can I reproduce \cite{zhang2019semantics} for German?
	\item Am I able to reach reported SOTAs of the data sets?
	\item Is there a difference for different head architectures? And if yes, why?
\end{enumerate}

\section{Thesis Structure}
In this first chapter \ldots\\
Chapter 2 introduces \ldots\\
Chapter 3 \ldots

