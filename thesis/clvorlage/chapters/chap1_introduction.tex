\newchap{Introduction}
\label{chap:1_intro}

% \epigraph{Not only the entire ability to think rests on language... but language is also the crux of the misunderstanding of reason with itself.}{\textit{Johann Georg Hamann}}

To begin with, the introduction will describe the intent of this MA thesis in the bigger context.
In order to do so, I will outline the general problems and topics of Natural Language Processing
and elaborate on the methods and techniques that were developed to address those questions. Then,
I will show how my approach ties in with current research and outline the aims and questions of my
thesis. The introduction will conclude with a short description of the structure of this thesis.


\section{Motivation}

Human language holds some truly mesmerizing features and puzzles, a lot of which are still not yet
understood in all their depth: For example, it is still unclear how children are able to learn the
grammar of their mother tongue from the corrupted and comparatively scarce language material they
are exposed to (cf. \cite{lust2006child, lewis2001learnability}).\myfootnote{The famous term describing this
phenomenon, ``Poverty of Stimulus'' (POS), was coined by \cite{chomsky1980cognitive}, who makes the case for
an innate language learning/processing faculty. This has led to a fierce debate over that issue:
Many researchers claim that the POS-motivated necessity for such an innate, human-specific,
genetic trait does not hold, and humans learn language by means of extremely sophisticated
statistical analysis. It has indeed been shown that infants are amazingly adept at extracting
statistical information from auditive input \citep{saffran1996statistical}. However, as was
countered by Chomsky and others, this does not fully explain the ability of children to apply
these statistical input cues to \emph{hierarchical} structures : ``The issue that is so central
to this particular POS problem is tacit knowledge by the child that grammatical rules apply to
such [hierarchical] structures'' \citep{berwick2013poverty} --- there is no behavioral explanation
for the tacit knowledge about hierarchical structures in the mind of the child. Therefore,
the POS argument still holds, although it has been somewhat weakened by empirical findings about
language acquisition driven by statistical insights.}

%(POS, Chomsky etc) Another astonishing fact about human language
% is the overwhelming amount of languages that exist today, even that number was probably much
% higher a few centuries ago.  As to how languages evolve, change over time and what trajectories
% of possible change may be, lots of questions are still open, and there remains enough work
% to do.

One of the most trivial and enigmatic traits about human language and communication is
that we actually \emph{understand} each other so well: That during a discourse, person
$X$ can reliably recover the encoded meaning of the expressions in the language signal
transmitted by person $Y$, and vice versa, especially given the ``noisy channel'' problem (cf.
\cite{plotkin2000language}), which is elaborated on a little more below. Further, we are
able to reconstruct a lot of information that is not explicitly stated in a verbal expression, and
uphold such a mutually shared belief of the state of affairs during a discourse. In fact,
this ``shared intentionality'' is hypothesized as being one of the traits that separates
human cognition and language from that of other animals (cf. \cite{tomasello2007shared}).

Because rational thinking, reasoning, and cognition in general are being perceived
as essentially linked to human linguistic competences, communicative abilities have been proposed
to act as a proxy for attesting intelligence to non-human entities. One of the most famous example
is the test proposed by Turing [1950]. Named after it's inventor, the "Turing-Test" is a method
for determining the capabilities of artificial intelligence systems: If the communicative behavior
of an artificial communicator is indistinguishable from that of a human agent, it is justifiable,
according to this view, to attribute ``intelligence'' to such a system.

However, reproducing human linguistic capabilities has turned out to be much
harder than was predicted during the 1950s, where scientists were enthused by
then recent developments of computers.\myfootnote{For instance, researchers of
the Georgetown University and IBM forecasted in 1954 that machine translation
would be a solved problem within a few years \citep{hutchins2005history}.} The
difficulty of reproducing human linguistic capabilities is mainly due to the
fact that humans are incredibly adept at handling the various complexities
present in the linguistic signal; It is with such unconcscious ease that we
process language, that we are rarely aware of how we handle complexities like
these:

\begin{description}
  \item[Vagueness] We have no problem dealing with vague statements like ``Most people never heard of it''; in a
    specific conversation situation we rely on extra-linguistic cues like pragmatics and common
    world knowledge to decide if ``most'' means 7.5 billion people, or 70\% of our friends, or if
    the approximate number is even relevant.
  \item[Ambiguity] The fact that many linguistic signs cannot be interpreted in only one way is a ubiquitous
    phenomenon in human language. This ambiguity can be observed on all levels of language: phonological, lexical, syntactic,
    semantic, pragmatic. A classic example for syntactic ambiguity would be the phrase ``He saw
    the elephant with a telescope''.
  \item[Corruption] Contrary to how we perceive language when speaking to each other, utterances are mostly
    not well-formed, grammatical sentences, but show a varying degree of stutter, incomplete
    phrases, repetitions,grammatical flaws, and other ``mistakes''. While not as present in
    written language, depending on the domain, there still is quite some noise present, e.g.
    in online chat threads ``Wers nocj wach??!!'' etc. Still, we mostly have no problems at
    all reconstructing the encoded information from such a corrupted signal --- contrary to
    algorithmic models, as will become evident during the course of this thesis[1~.
  \item[Common world knowledge] Normally the information we encode in ordinary conversation is highly condensed and as scarce
    as possible to enable swift conversation --- most of the actual information is reconstructed by the receiver, making use
    of general knowledge about the world (factual knowledge, such as ``Bern is the capital of
    Switzerland'' and ``Switzerland lies in Europe'', which together with logical reasoning, e.g.
    the \textit{modus ponendo pones}, leads to true deductions like ``Bern must be geographically
    located in Europe'') and the actual situation, time, and place the conversation takes place.
    A good counter-example would be legal texts, where every detail and implication of a judicial text
    must be tediously mentioned --- a circumstance which actually makes law documents nearly incomprehensible for the
    untrained reader.

\end{description}

So, every system that is built built with the aim of processing natural language in a ``deep'', human-like
manner must cope with this inherent fuzzinesses of human communication. In the field of
applied computational linguistics, often referred to as Natural Language Processing (NLP),
this subfield of research is known as Natural Language Understanding (NLU).\myfootnote{Note
that I will not engage in the discussion about whether it is philosophically appropriate
to claim that computational models ``understand'' human language. A lot of controversy has
arisen around this issue, with positions ranging from completely denying language models any
sort of (linguistic) understanding \citep{bender2021dangers} to the current trend of
concentrating on beating NLU SOTAs with ever larger models and blindly taking this as proof
of building architectures capable of learning human language. I would argue that the thruth
lies somewhere in between: Of course it is not enough to perform well on a standardized data
set to speak of ``understanding'', however, as \cite{sahlgren2021singleton} point out: Also
in humans, especially children, we measure language competences indirectly ``by using various
language proficiency tests, such as vocabulary tests, cloze tests, reading comprehension, as
well as various forms of production, interaction, and mediation tests \textelp{}''. Therefore,
I think it is not completely unsound to assess NLU capabilities of artificial systems
by measuring their performance on standardized datasets.}
NLU aims at building systems which are able to retrieve the semantic content encoded
in natural language and are able to further act upon it: For example, a chatbot should
be capable of ``understanding'' that the questions ``What's the weather like?'', ``Can
you tell me today's weather forecast, please?'', ``Will I need an umbrella today?'' all
have more or less the same meaning and should provoke the same answer and/or reaction
of the system.\myfootnote{Although one could argue that the third question differs from
the first two since it is a polar question and therefore a simple ``yes'' or ``no'' would be
grammatically correct, I would argue that one would perceive this as
a very dry, or even rude, answer and would expect a more elaborated answer in a regular
conversational context.}

% TODO: embed this quote


% ``It is always precarious to build arguments on inherently vague and general
% concepts such as `language', `understanding' and `meaning,' as the resulting
% theoretical constructs may become so overly general that they almost become
% vacuous.'' \cite{sahlgren2021singleton}

% \begin{quote}
%   learning a language entails learning the set of symbols and rules that define the
%   system. Learning the set of symbols equals vocabulary acquisition, while learning
%   the rules entails recognizing and formalizing grammatical, morphological, and
%   syntactic regularities.i We measure these competencies in humans — often indirectly
%   — by using various language proficiency tests, such as vocabulary tests, cloze
%   tests, reading comprehension, as well as various forms of production, interaction,
%   and mediation tests \textelp{}. \cite{sahlgren2021singleton}
% \end{quote}

% \begin{quote}
%   there is an overwhelming body of empirical results to demonstrate that current language models
%   have a passable capacity to detect the symbols and rules that define human language. This is not
%   what is under dispute in the current debate; what is under dispute is whether such structural
%   knowledge about the workings of the language system suffices.
% \end{quote}

\subsection{History, Methods, Problems of NLP and NLU}


% ``The `understanding' these models have of language is entirely structural,
% and thus does not extend beyond the language system'' \cite{sahlgren2021singleton}

% ``We can of course always question whether there is any `real' under-
% standing going on, but if the absence or presence of this `real' understanding
% has no effect on the behavior of the system, it will be a mere epiphenomenon
% that need not concern us.'' \cite{sahlgren2021singleton}

% ``consciousness is not an extra ingredient in addition to the complexity of a system:
% consciousness is the complexity of the system. Our point is that understanding is also
% not an extra ingredient of a symbol manipulation system: `understanding' is a term we
% use to describe the complexity of such a system.'' \cite{sahlgren2021singleton}

% ``Our point is that when the behavior of an NLU system becomes sufficiently complex, it will be
% easier to explain its behavior using intentional terms such as `understanding,' than to use a
% purely functional explanation.'' \cite{sahlgren2021singleton}

% The subsection of NLP that deals with the semantics, i.e. meanings, of utterances, is NLU.


NLP, and NLU as a sub-discipline of it, is ``[t]he engineering side of computational
linguistics'' that is ``largely concerned with building computational tools that do useful
things with language'' \cite{johnson2009statistical}. Therefore, the main goal of NLP is
not to describe or explain structures in human language employing computational theories,
but rather to solve specific problems regarding human language: For example, one might want
to design and implement a system that automatically detects the emotion that is transported
in a given text snippet. In order to approach these problems, researchers attempt to build
systems and models using different approaches.

In the relatively young history of the field of NLP, there are three phases identifiable
regarding the main architectural strategies of these models or systems, which I will briefly
outline:

During the first, or \emph{symbolic}, phase of NLP that lasted approximately from the 1950s
until the 1980s, systems addressing tasks such as machine translation were architectures that
consisted of carefully hand-written symbolic grammars and knowledge bases. This period was heavily
influenced by works in theoretical linguistics, especially Chomsky's transformational-generative
grammar \citep{chomsky2009syntactic,chomsky2014aspects}. The initial enthusiam over the theoretical
proficiency of such models, however, was soon severely dampened when it became clear
that the complexity of human language in real-world applications had been underestimated.


Beginning in the 1990s, a \emph{statistical turn} took place, and NLP turned towards
data-driven solutions, making use of the increased availability of large amounts of
electronic text on the one hand, and computers with increased computating power and
memory on the other \citep{liddy2001natural}. NLP problems were now being addressed
by learning patterns from huge data collections. One driver of this paradigm shift
were the various difficulties the traditional, symbolic systems entailed: Their
development ``requir[ed] a great deal of domain-specific knowledge engineering. In
addition, the systems were brittle and could not function adequately outside the
restricted tasks for which they were designed'' \citep[p.~13]{brill1997overview}.
The new, statistical {\color{red} approach} tried to tackle NLU-problems by shifting
the focus from tedious hand-crafting ``to empirical, or corpus-based, methods in
which development is much more data driven and is at least partially automated by
using statistical or machine-learning methods to train systems on large amounts
of real language data'' \citep[p.~13]{brill1997overview}. The main challenge for
engineers and scientists now lay in identifying suitable features and encoding them
in appropriate data structures, according to which the algorithm would learn helpful
patterns from the language data for solving the task at hand.

% A classical, although not really NLP specific,
% example would be spam detection: Instead of writing rules according to which an email gets
% classified as potential spam or not, one creates a dataset with thousands of mails, represented
% as bag-of-word vectors\myfootnote{The vocabulary $V$ consisting of all tokens in the emails
% gets encoded into a large vector $v^{|V|}$ where for each mail the dimensions corresponding
% to the words present in the mail} and each labelled if it is spam or not. a machine learning
% algorithm, e.g. a naive-bayes classifier, can now be trained on this dataset to ``learn'' the
% likelihood functions according to which an unseen mail then can by classified.

This orientation towards data-driven NLP solutions enabled new evaluation methods.
Using standardized data sets, different architectures could be compared to evaluate
their capabilities regarding the problem the dataset was constructed to display.

Half a decade ago, NLP and NLU entered a new stage: we are
now in the middle of the \emph{neural age} of computational
linguistics, where models are constructed implementing
deep, i.e. multi-layered, neural networks for learning
generalizable patterns and functions from data. For many
people in the NLP field, the advent of these machine learning
algorithms - powerful, but obscure and difficult to interpret
- also posed a threat to the established approaches regarding
NLP problems: ``Deep Learning waves have lapped at the shores
of computational linguistics for several years now, but 2015
seems like the year when the full force of the tsunami hit
the major Natural Language Processing (NLP) conferences''
\citep[p.~701]{manning2015computational}. Nevertheless, deep
learning approaches are an integral part of the vast majority
of NLP applications developed since 2015.

In contrast to the statistical period's main challenge --- the identification
and extraction of suitable features ---, the algorithms in the current, neural
approaches are learning the features that are most informative for a given task
from the raw data - on their own.\myfootnote{This is probably most obvious in
computer vision contexts, where the input data for neural models is the raw RGB
pixel image data, in contrast to pre-deep learning algorithms, where the input
of a machine learning system would be a data structure produced by some feature
detection algorithm, such as SIFT \citep{lowe1999object}. For text, however,
the case is not as clear as to what the ``raw'' representation of it for a
model would be, as will be shown in the next chapter.} The human part in the
process is now ``reduced'' to designing the overall model architecture, defining the
hyperparameters of the model and the training set up, and compiling large enough
amounts of data that are, in the best case, also of good quality.

While the roughly sketched methods above apply to a wide range of applications in NLP, I will now
point to some of the inquiries NLU aims at: Simply put, NLP is concerned with the structural side
of natural language text, while NLU looks at the content of these utterances. For example, typical
NLP tasks, such as dependency parsing, POS-tagging, and coreference soltution do not require a
semantic representation of words or phrases --- often, it suffices to analyze structural properties
such as morphology, unsophisticated frequency statistic, or transition probabilities to solve such
problems to an acceptable extent (cf. \cite{kumawat2015pos}). By contrast, NLU tries to process
language in a manner similar to humans: We infer something from language, reconcile communication
information with our world knowledge, answer questions, detect logical inconsistencies etc. NLU is
the endeavor of building artificial agents that are able to do these things as well.

% \cite{mcshane2017natural} in her paper on NLU in cognitive agents argues that truly
% NLU-capable systems's abilities must go beyond ``mere'' symbol processing:

% \begin{quote}
%   cognitive agents must be nimble in the face of incomplete interpretations since even people do not
%   perfectly understand every aspect of every utterance they hear. This means that once an agent
%   has reached the best interpretation it can, it must determine how to proceed — be that acting
%   upon the new information directly, remembering whatever it has understood and waiting to see
%   what happens next, seeking out information to fill in the blanks, or asking its interlocutor
%   for clarification. The reasoning needed to support NLU extends far beyond language itself,
%   including, nonexhaustively, the agent’s understanding of its own plans and goals; its dynamic
%   modeling of its interlocutor’s knowledge, plans, and goals, all guided by a theory of mind;
%   its recognition of diverse aspects of human behavior, such as affect, cooperative behavior,
%   and the effects of cognitive biases; and its integration of linguistic interpretations with
%   its interpretations of other perceptive inputs, such as simulated vision and nonlinguistic
%   audition. Considering all of these needs, it seems hardly possible that fundamental NLU
%   will ever be achieved through the kinds of knowledge-lean text-string manipulation being
%   pursued by the mainstream natural language processing (NLP) community. Instead, it requires
%   a holistic approach to cognitive modeling of the type we are pursuing in a paradigm called
%   OntoAgent.
% \end{quote}

However, all disagreements about the very nature of ``understanding'' and ``communication'' set aside,
there probably is consensus that the field is far from being able to construct a system that
would pass the Turing test, i.e. an agent which communicative abilities are on par with a human being.
Therefore, the strategy is to analyze the general human communicative ``understanding'' capability
and identify core building blocks of it, which are then addressed in isolation.



\subsection{Subtasks of Solving NLU}

As the endeavor of building a holistic, fully NLU capable agent is probably an objective
too ambitious to achieve in one go, the strategy in modern computational linguistics is
to break it down into smaller ``subproblems'': Several core building blocks of NLU have
been identified and datasets comprising examples tailored at those have been compiled.
The performance of new models and approaches on such data sets is often taken as a proxy
for attesting NLU capabilities targeted at the task.

Some of these proposed subproblems or proto-tasks of NLU are the following:

\begin{description}
  \item[Sentiment Detection] Given a sentence, we can normally assess if the emotion transmitted by
      it is rather positive, negative, or neutral: ``Oh my, what a lovely day!'' is clearly positive,
      while  ``asdasd'' probably is not.
  \item[Grammaticality Recognition] Being able to understand utterances in a language also implies
      being able to judge the grammaticality of any statement in that language: ``Eagles that fly can
      eat'' is judged as valid by English speakers while ``Eagles that fly eat can'' is not.
  \item[Entailment Recognition] An important ability of understanding is the recognition of the
      logical relationship in which two utterances stand. In other words, given two sentences $A$ and
      $B$, does $B$ follow from $A$? For example: Given the sentence ``The weather forecast predicts
      rainfall the next days'', does the sentence ``Tomorrow will be a beautiful, sunny day'' conform,
      contradict or stand in a neutral relation to the first sentence?
  \item[Question Answering] To locate a (or rather, the) relevant text span in a given context according to some
      questions is also a task where one would assume that some sort of understanding is needed:
      Given the question `` What was the name of the King of England?'' and the context ``Henry
      V (16 September 1386 – 31 August 1422) was King of England from 1413 until his death in
      1422.'', a system would would need to extract the span ``Henry V''.
\end{description}

Since each of these tasks and datasets are focused on a specific subpart of general NLU, it
is by no means implied that a system which e.g. performs well on entailment recognition will
be able to extract answer spans to an acceptable degree. Therefore, to assert that some model
or approach does indeed capture some properties of general NLU, it has become standard to
evaluate models on several of those subtasks. One of the best established compilations of such
subtasks is the GLUE benchmark \citep{wang2018glue}. GLUE, which stands for \textbf{G}eneral
\textbf{L}anguage \textbf{U}nderstanding \textbf{E}valuation, is `` a collection of NLU tasks
including question answering, sentiment analysis, and textual entailment, and an associated
online platform for model evaluation, comparison, and analysis.'' \citep{wang2018glue}.

Since this thesis is concerned with German and there is no such standardized benchmark, I
compiled one myself --- GerGLUE. GLUE and GerGLUE are described in more detail in
chapter \ref{chap:3_datasets}.


\subsection{Text Representations}

Text can be analyzed and represented in various ways; in fact, those characters on the sheet
you are reading right now are one form of representing language. However, for an algorithm,
language represented by means of an alphabetical script is not a well-suited encoding ---
since de Saussure's lectures at the start of the 20th century, it is established that the
actual characters are completely arbitrary in what they denote \citep{de1989cours}. There
is nothing in the properties of the character(-sequence) \emph{itsef} which would relate it
necessarily to the thing it represents. Nothing in the mere shape of the character sequence
``contrary'' implicates on what meaning, or signifié, this string, or signifiant, maps onto.
As human beings, if capable of reading the Latin alphabet and knowing the English language,
we decipher the string to symbolize the linguistic concept, we pronounce /\textprimstress
k\textscripta nt\textturnr \textschwa \textturnr i /.\myfootnote{There exist so-called
featural writing systems which actually encode phonological information in the form of
their characters, e.g. the Korean Hangul alphabet, which encodes the position of the tongue
for producing a consonant \citep{sampson1985writing}.} In other words, for a computational
system to act upon human language input, we must find a way to encode this input in a way
that actually represents structural information; depending on what the model is aimed for,
this may be phonological, syntactical, or semantic properties.

% \myfootnote{For simplicity, I will
% focus on written language in my thesis, however this applies to other modalities like
% speech and signing as well.}
For example, one could represent each word of a sentence as a number, e.g. the page number
of the Oxford English Dictionary on which the definition of the word --- or its lemma base
form ---is given: <Every> <event> <has> <a> <cause> $\rightarrow$ <234> <229> <388> <12>
<176>. However, that would probably not be very informative. But if we represent each word
of a sequence by its Part of Speech, a computational system designed for e.g. identifying
Named Entities might retrieve task-supporting information from it --- especially if both
representations, the ``normal'' one and the POS one, are combined: <Every DET> <event NOUN>
<has VERB> <a DET> <cause NOUN>.


\subsubsection*{Embeddings}


Most NLU models don't operate on the text as it is, i.e. as an array of UTF-8 encoded
signs. Often, it is easier to implement an algorithm that processes text in some way, to
encode this text numerically. A strategy adopted quite early is the so-called Bag of
Words \citep{harris1954distributional} technique, which encoded a sequence of words $s$
in one vector, basically indicating what items of a given vocabulary are present in $s$.
% Given a corpus of texts, the whole vocabulary $V$, or some subset,
% e.g. the most frequent 10,000 tokens, gets encoded in one large vector $v \in 0^{|V|}$. Where
% the $i$th dimension of $v$ stands for the $i$th item in $V$.
% To encode a sentence $S$ in this vector, for each token in it, the correlating dimension of the
% vector gets set to 1.
While this BOW technique is quite effective for certain tasks, e.g. information extraction
systems often make use of it, it has some flaws: It fails in particular to reflect the core
property of language being inherently sequential --- the BOW encodings of ``Alice hit Bob''
and ``Bob hit Alice'' are indistinguishable.

Therefore, other methods of representing word sequences numerically have been devised, which assign
to each word in a sequence a numeral representation, thus preserving the sequential information.
Examples of such a representation would be Latent Semantic Analysis \citep{furnas1988using}, which
generates word vectors by reducing high-dimensional word co-occurrence matrices via singular value
decomposition, or word2vec \citep{mikolov2013distributed}. This opened the possibility of letting a
neural model compute those embeddings, while applying a language modeling objective.


\paragraph*{Contextualized Word Embeddings}

Research on the generation algorithms of such numerical word embeddings, which can
be used as text representations for various NLP tasks, has experienced tremendous
interest in recent years, producing various architectures and frameworks, leading
to the current state-of-the-art: contextualized word embeddings. Contextualized word embeddings
account for the fact that the semantic function of a word is heavily dependent on
the context in which it appears. For example, the word ``building'' has quite different semantic
and syntactic functions in the following sentences: ``This is a very high building!''
and ``She was building an empire'' --- in this obvious example\myfootnote{The two
sentences exemplify the problem of homonymy: Two different lemmata having the same
word form. However, even if the same lemma is used in different context, subtle
changes in semantics and grammatical function are present.}, a word representation
framework which does not account for the context, will clearly fail to adequatly interpret
such sentences.

The afore mentioned neural approaches in NLP have proven to be very adept at
producing such contextualized word representations: ``The gains so far [from Deep Learning] have
not so much been from true Deep Learning (use of a hierarchy of more abstract
representations to promote generalization) as from the use of distributed word
representations—through the use of real-valued vector representations of words
and concepts.'' \citep[p.~703]{manning2015computational}

% CONTUNIATION OF QUOTA: Having a dense, multi-dimensional representation of similarity between all
% words is incredibly useful in NLP, but not only in NLP.

One effective framework of producing such contextualized word embeddings is
BERT \citep{devlin2018bert}, which I also employ for the experiments in my
thesis. The general concepts of contextualized word embeddings, transfer
learning, and BERT in particular will be described in depth in the following
chapter \ref{chap:2_approach}, section \ref{sec:pretrained-embeddings}.


\subsubsection*{Semantic Structures}

While the motivation for representing a sentence numerically is to let algorithms operate
``autonomously'' on these representations, for example by computing some distance measures
between a query and a set of texts for information retrieval, one can also try to encode
some structural information of the sentence at hand. To support a model with information
about what relations are present between different parts of a sentence, one could provide
the algorithm with the syntax parse tree of the sentence. This way, the difference between
the two "hitting"-sentences mentioned earlier would be clearly distinguishable.

\paragraph*{Semantic Role Labels}

Semantic Roles are a linguistic tool developed for analyzing a sentence with regard to the
semantic relations that hold between different entities involved in the action described by
the sentence, thus ``explaining both the syntactic structure as well as the meaning of sentences''
\cite{bussmann2006routledge}. It is sometimes considered to be one of the oldest concepts in
linguistic, conceptually contemplated as early as in the 6th century BCE by the old-Indian
grammarian P\=a\d{n}ini (cf. \citep{gildea2002automatic}).

The core intention is to identify generalizable semantic functions, or semantic roles,
that participants in an event can engage in. With such an instrument, it is possible to
model the semantic equivalence of grammatically and syntactically quite distinct sentences
(after \cite{palmer2010semantic}):

\begin{examples}
  \item John broke the window.
  \item The window broke.
\end{examples}

In the first sentence, ``break'' is a transitive verb, while it is intransitive in the second.
``Window'' is the grammatical object in prior, while being the grammatical subject in the latter
sentence. Despite these differences, the expressed action, or the encoded information, is the
same both times: an object (in the physical, not grammatical sense), the window, was shattered.

Semantic Roles are an attempt to formalize the semantic content of sentences by attributing each constituent
a generalizable label. In this case, ``the window'' would be labelled as \emph{patient}
, i.e. the participant undergoing a change of state, in both sentences.
In the first, where there is also a clear initiator of this change, ``John''
would be labelled as the \emph{agent} of the event.

Thus, as a further form of representation, any sentence could be represented by the words in it
replaced by their Semantic Role Label (SRL): <John> <broke> <the> <window> $\rightarrow$ <AGENT>
<PREDICATE> <PATIENT> <PATIENT>. (Or, as mentioned before, a combination of the ``normal'' text
representation combined with SRLs)

A more detailed elucidation on the concept semantic roles and its history is provided in chapter
\ref{chap:2_approach}, section \ref{sec:semantic-roles}.



\section{Contributions/Goals}

The goal of this thesis is to explore if a combination of the two text representation
techniques described above --- Contextualized Word Embeddings and Semantic Role Labels ---
may lead to embeddings which support machine learning algorithms targeted at NLU tasks. For
English, \cite{zhang2019semantics} showed that combining BERT embeddings with numerically
encoded SRL information leads to an increase in performance on NLU tasks, compared to the
vanilla\myfootnote{In computer science and hacker culture in general, ``vanilla'' is the term
describing the use of computer hard and/or software without customization of its original
implementation (cf. \href{http://www.catb.org/~esr/jargon/html/V/vanilla.html}{The Jargon
File}).} BERT architecture. However, their findings are somewhat inconsistent and, probably
due to the format of a short paper, the description of the architecture and sound analysis
of the results are rather meager.\myfootnote{For example, it is not clear \emph{what}
results for the datasets are being reported: The best run out of several experiments, the
average performance of several experiments, etc. (due to the random initialization of the
weight matrices of the head modules and SRL encoder, there are expectable fluctuations in
performance, which makes it highly improbably that only one model per task was trained).
Neither control \citeauthor{zhang2019semantics} for statistical significance, which makes
sound statements about architecture superiority quite disputable.}

Although my approach stands in the tradition of recent machine learning trends in
NLP --- i.e. training a model on data and using test data to asses its capabilities
--- I nevertheless also view it as an attempt at re-integrating linguistically
motivated structures into NLP, as the linguistic nature of NLP is sometimes left
on the sideline. As \cite[p.~702]{manning2015computational} observes: ``Recently
at ACL conferences, there has been an over-focus on numbers, on beating the state
of the art. Call it playing the Kaggle game.'' Despite the apparent success of
linguistic-agnostic architectures on NLP benchmarks, those models tell us little
about the nature of human language and the structural properties of it. To continue
with \citeauthor[p.~706]{manning2015computational}: ``It would be good to return
some emphasis within NLP to cognitive and scientific investigation of language
rather than almost exclusively using an engineering model of research.''

By implemeting textual representations that are explicitly grounded in linguistic theory, I
see part of the contribution of my model in antagonizing the purely data-driven score-games of
modern NLP. I hope to provide an approach that deviates from the brute-force data engineering of
many current research projects, which makes use of the affordances of contemporary computational
linguistics while being motivated by linguistic concepts and established frameworks and the
possibility of gaining insights into the inner workings of human language.


\section{Research Questions}

This thesis sets out to answer the following research questions:

\begin{enumerate}
  \item \label{ques:one} Does the combination of  BERT embeddings with structured SRL information have a positive,
                         measurable effect on NLU tasks?
    \begin{itemize}
      \item Since I replicate, in some sense, the work of \cite{zhang2019semantics}, are my findings similar to what they report?
      \item Are there differences between datsets, registers, tasks, etc.?
      \item Are the datasets well-suited for making sound statements about the effect of enriching BERT embeddings with SRLs?
      \item If the preceding question has to be answered negatively, does not hold, is there a way to determine nevertheless if SRLs might have a positive effect in more appropriate settings?
      \item Can I determine what aspect(s) of SRLs support models in downstream tasks?
    \end{itemize}
  \item \label{ques:two} Which head architecture for fine-tuning BERT-based, SRL-enriched text representations is best suited for
                          NLU tasks?
  \item \label{ques:three} What are the effects of different representation methods of combinations of BERT-subtokens with SRLs and different encoding techniques on the multilayered nature of semantic roles?
    \begin{itemize}
      \item The number of considered predicate-argument
            structures is fixed to be three per sentence --- is it more effective to fill empty slots with
            0-SRLs, or duplicating existing SRLs?
      \item To merge BERT embeddings with SRLs, either
            the splitted BERT subtokens need to be merged back or the SRLs have to splitted up --- does
            this lead to different results?
    \end{itemize}

\end{enumerate}



\section{Thesis Structure}

In this first chapter I gave a very brief overview of some general problemse and proposed solutions in
NLP over the past decades and highlighted the critique recent purely data-driven approaches have been
receiveing from the linguistic fraction of NLP.

Chapter \ref{chap:2_approach} introduces the basic concept of BERT and its shortcomings,
and presented solutions or improvements. Further, I explain my approach and the relating
linguistic concepts.

The datasets I compiled in my GerGLUE collection to test my models on are described in detail in
chapter \ref{chap:3_datasets}.

In chapter \ref{chap:4_architecture}, I describe the architecture of my GliBERT model in all
detail: The identification and encoding of the semantic role labels, the different combination
procedures of the BERT embeddings with these, as well as the different head architectures.

The overview and discussion of the performance of the various model architectures on
the GerGLUE data set is made in chapter \ref{chap:5_results}.

Finally, I draw some conclusions and summarize the gained insights in the last chapter \ref{chap:6_conclusion}.

