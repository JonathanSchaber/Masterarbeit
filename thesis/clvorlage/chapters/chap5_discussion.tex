\newchap{Discussion}
\label{chap:5_discussion}


Based on the results presented in the previous chapter \ref{chap:5_results}, A first, preliminary
insight can be formulated: Adding numericall embedded SRL information to the German  BERT embeddings
leads to overall enhancing in classification tasks. For question answering, however, the SRL information
leads to a clear decrease in performance. Additionally, while there was no clear advantage of one aligning strategy
over the other for the classification tasks, the situation for question answering is crystal clear: Preserving
the BERT subtokens and duplicating SRLs clearly outperform the subtokens merging approach.

The reasons for these rather drastic differences between classification and question answering
tasks remain unclear to me; especially the negative impact of SRLs for question answering tasks
--- despite tediously controlling the merging algorithm and validating the re-computing of start
and end indices, I could not make out any error.

My speculation towards an explanation would be, that one of the main differences between
the classification and question answering datasets is the size of the examples: While for
the classifications task the model needs to extract semantic information out of a handful
sentences, for the question answering tasks this is rather the exception --- normally
the context from which the answer span must be extracted is multiple sentences long. My
suspicion is that SRL information can highlight some crucial semantic relations present in a
sequence which would be hard for the vanilla BERT to extract; maybe because it is in marked
passive voice, not obviously negated etc. However, for extracting answer spans out of a long
text sequence consisting of several sentences, each having potentially 3 predicate-argument
structures present, the SRL information may not display a helpful signal to the head, which
then relies more on information present in the vanilla BERT subtokens. This hypothesis is
further substantiated by the analysis of BERT-SRL combination strategy superiority, depicted
in table \ref{tab:qa-gain-loss-token-merged}: Maintaining the BERT inherent subtokenization
structure outperforms the merging approach by far, indicating that the head relies heavily
on information encoded in the BERT subtokens.

For the classification tasks, there was a modest tendency in the results suggesting that
SRL information is of more support in sentence pair tasks, compared to single question datasets.
However, this conjecture needs also to take into account that the register in the sentence
pair tasks was generally of a higher standard than in the single sentence tasks; the
extreme being SCARE, with its colloquial,

One might ask if a simple accuracy is the right metric to make this substantial claims
about model performances and the utility of SRL-enriched embeddings. Moreover, it is
often common to compute other metrics like F1-scores, especially if class imbalance
is a known property of datasets. There are two main reasons as to why I chose to stay
with simple accuracy: (1) I am not interested in building a classifier which explicitly
addresses problems related to failure avoidance or class discrimination, e.g. spam
detection tasks. In such cases, an F1-score of course is more meaningful because it
is more sensible to predictions on underrepresented but task-important classes (such
as spam). In this thesis, however classification performance without focusing on a
particular class is of interest. (2) As reported in chapter \ref{chap:3_datasets}, most
of my datasets do not show signs of class imbalance, and their ``examples are assumed
to be identically and independently distributed (IID)'', in which case the accuracy measure is
``the most used empirical measure'' \citep{sokolova2006beyond}. To further substantiate
this, I compute some metrics for the most significantly imbalanced dataset, namely
SCARE: A brute force ``stupid'' prediction of the majority class on the test set would
results in an accuracy of 59.09\% and a macro F1 of 24.76. The example confusion
matrix of an actual experiment is shown in table \ref{tab:confusion-scare}: Although
the algorithm overpredicts the majority (``Positive'') class, it also predicts to some
extent the other ones. The measures resulting from this experiment are an accuracy of
83.71\% compared to the macro F1 of 73.52. Therefore, although the accuracy may distort
the picture to some extent, this is negligible in the context of this thesis.


\tab{tab:confusion-scare}{Confusion matrix for SCARE $\alpha$ +SRL merged duplicated [CLS] head.}{
  \begin{tabular}{|ll|ccc|}
    \hline
                                                    &           & \multicolumn{3}{c|}{Predicted} \\
                                                    &           & Positive & Neutral & Negative  \\ \hline
    \multirow{3}{*}{\rotatebox[origin=c]{90}{True}} & Positive  & 149      & 12      & 5         \\
                                                    & Neutral   & 4        & 11      & 6         \\
                                                    & Negative  & 3        & 8       & 66        \\ \hline
  \end{tabular}
}{Confusion matrix for one SCARE +SRL ensemble}

\section{GliBERT Noise Nuisance Analysis}
\label{sec:glinoise}


After reporting the general results, aggregations and the somewhat disenchanting conclusions
regarding the positive effects of enriching BERT embeddings with SRLs, I now try to investigate
the reasons for this mediocre effect. After investigating the data and SRL quality, I found that
noise on several levels is present which probably led to too much randomness in the complex
functions the models needed to approximate. I classify the observed data irregularities into the
following \textbf{noise} categories:


\begin{itemize}
  \item[\textbf{register}] The textual styles vary greatly from utilizing complex,
                           hypotactic sentence structures (e.g. XQuAD), to highly
                           informal, elliptic --- even erratic --- structures
                           (e.g. SCARE).
 \item[\textbf{dataset}] Many of my datasets were constructed either automatically
                         (e.g. scrambling text automatically to create paraphrase pairs)
                         or employing crowd-sourcing techniques. Either way, the process
                         is prone to errors. There are, e.g., 84 sentence pairs in the training
                         set of PAWS-X that are 100\% identical, yet labelled as non-paraphrases.
 \item[\textbf{translation}] Due to the mostly employed semi-automatic translation
                             approach for creating the various datasets, errors
                             have been introduced into the data ranging from typical
                             translation errors (e.g. English ``bishop'' in the clerical
                             context translated to the German chess figure counterpart
                             ``Läufer'', not ``Bischof'') to eventually wrongly copied
                             labels, since the overall meaning changed during
                             the translation process (e.g. a sentence pair is no more
                             contradicting but neutral), thus creating label noise.
 \item[\textbf{SRL}] The SRLs obtained from DAMESRL are, conservatively formulated,
                     questionable in their quality (e.g. modifiers are completely missing).
\end{itemize}

While it is normal for data in NLP to possess noise in some way or the other, this usually does
not impact the overall performance of a model trained on it in a severe way, since information
in language is encoded redundantly, and noise present in one channel can be compensated for by
non-corrupted information from another channel.\myfootnote{For example, information about which
constituent takes the syntactic function of the subject is often not marked by using only one
possibility, e.g. though a case ending. Often, several strategies are present, as in German: The
subject is marked morphological through case endings (which are not always salient), positional
(in unmarked contexts), and structurally (in unmarked contexts, normally the proto-agent is
realized as subject) (cf. \cite{bussmann2006routledge,jaeger2010redundancy})} However, if noise
is present on too many channels at the same time, reliably recovering structural information may
break down to some extent.
% In short --- the good old GIGO concept from informatics holds mutatis
% mutandis also in NLU, and NLP in general.



\subsection{Register Noise}
\label{sec:register-noise}

German BERT, according to deepset, was pretrained on the German Wikipedia dump, the OpenLegalData dump and
news articles. All of this corpora comprise grammatical, to the standards of orthography adhering text
belonging to a rather formal, ``professional'' register. Three of the six datasets in GerGLUE, namely
deISEAR, SCARE, and XNLI, however, can be described as representing a rather colloquial register of
language, even displaying features of transcribed speech as e.g. in XNLI:

\begin{examples}
  \item Ja und wir haben es irgendwie behalten, äh irgendwie äh, es war eine Überraschungs-Geburtstagsparty für sie, das war es Sie liebte die Überraschungsfeier.
\end{examples}

Naturally, since the German BERT module probably wasn't confronted with texts coming from such a register,
the BERT embeddings will probably contain some noise regarding the classification task at hand. Also, ParZu
is aimed at parsing ``clean'', correct sentences, showing of course problems at processing examples as the
one above, which leads to the known error propagation and amplification through DAMESRL.

In other words, it is expectable for GliBERT to show difficulties on the aforesaid datasets, which
was also observed during the analyses of the results in section \ref{sec:classification-results}.

Interestingly, the discrepancy between the German BERT pretraining text register and the text
present in some GerGLUE datasets, the analysis of the German BERT tokenization statistics shows
that also in those problematic datasets, there are very little out of vocabulary items, which
would be encoded by the catch-all [UNK] BERT token:

\fig{images/all_data_sets_token_stats.pdf}{fig:all-data-sets-token-stats}{Percentages of token-types in all datasets.
                                                                          \#\# Subtokens represent the amount of tokens that
                                                                          get re-merged in the merged settings (e.g. ``Master''
                                                                          ``\#\#arbeit'' $\rightarrow$ ``Masterarbeit''. The amount
                                                                          of tokens that lie outside of the German BERT vocabulary is
                                                                          in all datasets extremely small (for deISEAR and XNLI
                                                                          there are no [UNK]s at all); the largest shares of such tokens
                                                                          are present in MLQA and XQuAD with .79\% and .73\%,
                                                                          respectively.}{14}{Token Types all Datasets}



\subsection{Data Set Noise}
\label{sec:label-noise}

As \citep{caswell2021quality} point out, apparently there is often a lack of
prudence observable when multilingual datasets get compiled --- especially for
low resource languages they report devastating discoveries of ungrammatical
and even non-sense texts for these.


However, especially for sentence pair datasets, one of the sentences needs to be constructed. This
can happen by automatically construct these sentences (PAWS-X) or let humans come up with those
sentences (as for XNLI). In both cases, there might happen errors, and the label assigned to a
sentence pair may be actually wrong. This is obvious for PAWS-X, where e.g. in the training set,
there are 84 identical sentence pairs are labelled as non-paraphrases, which is obviously false,
since the other 3,125 exact identical sentence pairs are labelled as paraphrases. E.g. sentence pair
number 45061 (in the original English PAWS and the German PAWS-X datasets), both labelled as
non-paraphrases:

\begin{examples}
  \item Riverton was a parliamentary electorate in the New Zealand region of Southland .\\
        Riverton was a parliamentary electorate in the New Zealand region of Southland .

  \item Riverton war ein Parlamentswähler in der neuseeländischen Region Southland.\\
        Riverton war ein Parlamentswähler in der neuseeländischen Region Southland.
\end{examples}

But also for human created sentence pairs, there
is a chance that the gold labels are not that clear. Further, through the creation of the German
datasets by means of semi-automatically translating English datasets, additional translation
noise might additionally pollute the data.

To examine this more systematically for XNLI and PAWS-X, I conduct a re-labeling of 20 random
examples each  where the predictions of the best model contradicted the gold labels in the dataset.


\subsubsection{Re-annotation}

\paragraph*{PAWS-X}

For the 20 re-annotated examples in PAWS-X the Fleiss' $\kappa$ of 0.68 indicates ``substantial'' agreement
between the two human annotators; however, if the coefficient is computed between the two human annotators
and the gold labels of those 20 examples, the value drops to 0.35, indicating substantive disagreement
as to what sentence pairs are to be considered paraphrases and which don't.

Even more interestingly, in 30\% of the considered cases, the model and the human annotators unanimously
disagree with the gold labels present in PAWS-X:

% 2 human annotators re-label 20 examples of PAWS-X where gold != predicted.
% Fleiss' $\kappa$ between 2 annotators: 0.68
% Fleiss' $\kappa$ between 2 annotators and gold: 0.3541
% Fleiss' 4\kappa$ between all $-$0.07

% 6 examples where both annotators agree with predictions, disagree with gold:

\begin{examples}
  \item Der NVIDIA TITAN V wurde von Nvidia am 7. Dezember 2017 offiziell angekündigt.\\
        Am 07. Dezember 2017, verkündete NVIDIA offiziell Nvidia TITAN V.

        humans \& model: False, Gold: True
  \item Die Schäfte sind sehr kurz oder oft nicht vorhanden.\\
        Es sind entweder wenig Landschaften vorhanden oder sie fehlen in den meisten Fällen.

        humans \& model: False, Gold: True
  \item 1963 trat Roy der Kommunistischen Partei Indiens bei und leitete Gewerkschaftsbewegungen in Bansdroni in Kalkutta.\\
        Roy trat 1963 der Kommunistischen Partei Indiens bei und leitete Gewerkschaftsbewegungen im Kolkata-Gebiet von Bansdroni.

        humans \& model: True, Gold:False
  \item Der Kanal ist einer der ältesten schiffbaren Kanäle Europas und sogar Belgiens.\\
        Der Kanal ist einer der ältesten befahrbaren Kanäle in Belgien und Europa.

        humans \& model: True, Gold:False
  \item Propilidium pelseneeri ist eine Art der Meeresschnecken, eine wahre Napfschnecke und Gastropoden-Mollusk in der Familie der Lepetidae.\\
        Propilidium pelseneeri ist eine Art der Meeresschnecken, eine wahre Napfschnecke und Meeres-Gastropoden-Mollusk der Familie der Lepetidae.

        humans \& model: True, Gold:False
  \item Die Chicago Bears sanken auf die Giants 27:21, und verloren 0:6 zum ersten Mal seit 1976.\\
        Die Chicago Bears verloren 21:27 gegen die Bears und standen erstmals seit 1976 bei 0:6.

        humans \& model: False, Gold:True
\end{examples}

In all of this cases it seems to be the case that the given gold labels actually are false,
leading to the conclusion that the evaluation and measured performance on such a dataset must
be taken with caution, and not just blindly accepted as ultimate truth.


{% PAWS-X; different repair-strategies $\rightarrow$ different labels (gold: false)

% Sawyers autorisierte Biografie wurde 2014 von Huston Smith veröffentlicht.
% Im Jahr 2014 wurde Huston Smith eine autorisierte Biographie von Sawyer veröffentlicht.


% Im Jahr 2014 wurde {\color{red} «}Huston Smith{\color{red} », } eine autorisierte Biographie von Sawyer{\color{red} ,} veröffentlicht.

% Im Jahr 2014 wurde {\color{red} von|für|durch|trotz|wegen} Huston Smith eine autorisierte Biographie von Sawyer veröffentlicht.

% Im Jahr 2014 wurde Huston Smith eine autorisierte Biographie von Sawyer veröffentlicht.


\paragraph*{XNLI}

For XNLI, where in contrast to PAWS-X each example can have one of three labels instead of
binary classification, the findings differ to some extent: While the agreement only between
the two re-annotators is measured as a low Fleiss' $\kappa$ of 0.36, it raises to 0.48
when the gold labels are also taken into account, indicating ``moderate agreement''. For
one example the humans and model were on the same page regarding the label but disagreed
with the gold label:

% 2 human annotators re-label 20 examples of XNLI where gold != predicted.
% Fleiss' $\kappa$ between 2 annotators: 0.36
% Fleiss' $\kappa$ between 2 annotators and gold: 0.4787
% Fleiss' 4\kappa$ between all 0.08

% 1 example where 2 humans == model and 2 humans != Gold

\begin{examples}
  \item Bato ist ein Jahrhunderte altes Wort, das man als Kerl oder Kumpel übersetzen kann.\\
        Bato (oder Vato) ist ein spanisches Wort, das Typ oder Typ bedeutet.

        humans \& model: neutral, Gold: Entailment
\end{examples}

% 1 example where 2 humans != model and 2 humans != Gold
Since the specification of ``spanisch'' is only encountered in the hypothesis but missing from the premise,
I would argue that the humans and the model are right about the ``neutral'' label and the gold label
is false.
In another example, all three ``agents'' --- the two humans, the model, and the gold labels --- disagreed:

\begin{examples}
  \item Oh, ich sehe oh der Staat braucht es nicht gut, das ist eher das, das ist eher ungewöhnlich, nicht wahr?\\
        Das macht Sinn, dass der Staat es benötigt.

        humans: neutral, model: entailment, Gold: contradiction
\end{examples}

The latter is also a further exemplification of the register diversity in XNLI: The premise above seems as if it
was a transcription of an utterance torn out of the context an actual conversation; because of that for a human
reader there is essential information missing, making it a rather strange and vague case for attributing
a specific entailment regarding any hypothesis.


Both re-labelings have led to the finding that the quality of semi-automatically
created datasets is often doubtful: (1) As the many ``neutral'' re-labelings of
XNLI examples by the human annotators indicate, often there seems not be a clear
preference towards a certain label --- rather, as in the examples above, there is
too much ambiguity and lacking of needed information around to make a clear decision
of the relation between those sentences. (2) Further, the predefined labels in such
datasets are by no means to be taken as undoubtedly indicating true classifications.



\subsection{Translation Noise}
\label{sec:translation-noise}

Automatic translation has come a long way and systems nowadays produce often qualitatively
high results --- at least for translations between high resource languages, such as English
and German. But still, especially for delicate, subtle expressions, errors can happen and
(partial) mistranslations are being produced. While this potentially leads to translation
artifacts, which may confuse the model, there also exists the possibility of additionally
introducing label noise.

Take for example the following XNLI English original premise-hypothesis pair and its aromatically
translated German counterparts:

\begin{examples}
  \item and that's a lot of it is due to the fact that the mothers are on drugs\\
        The mothers take drugs.

  \item Und vieles davon liegt daran, dass die Mütter Medikamente nehmen.\\
        Die Mütter nehmen Drogen.
\end{examples}

The English term ``drug'' indeed is an ambiguous word, which either can refer to a pharmaceutical
product that is used as medicine or to an abusively taken substance (which may be of the first
type).\myfootnote{\url{https://dictionary.cambridge.org/dictionary/english/drug}} However, since
the label of the original sentence pair is \emph{entailment}, the term would have needed
to be translated in the same meaning both times. However, this did not happen, and thus
a wrongly labelled training instance was created.



\subsection{SRL Noise}
\label{sec:srl-noise}

A major question arising in the context of using automatically assigned Semantic
Roles in downstream tasks, is how precise these predicted Semantic Role are. Since
there is no gold standard available for Semantic Role Labels for the datasets I
use in my experiments, there is no straight-forward way to evaluate their quality
automatically. Additionally, in contrast to other tagging tasks like POS prediction
or NER, Semantic Roles are not as black and white: While it is relatively easy to
decide if a predicted POS tag is correct or incorrect, it is more a scalar value
concerning the correctness of SRLs; often, a semantic role labelled squence is not
completely ``false'' but rather not really informative or precise.

\begin{wrapfigure}[19]{l}{0.65\linewidth}
  \begin{center}
    \includegraphics[width=0.9\linewidth]{images/SRL_assessment.pdf}
  \end{center}
  \stepcounter{myfigure}
  \caption[SRL assessment]{Independent evaluation of SRL quality by three people. Regardless of the
    label attributed to each example, it is obvious, that the total amount of sentences for which the
    annotators evaluated the corresponding semanti roles as \emph{helpful}, is relatively stable.}
    \label{fig:SRL-assessment}
\end{wrapfigure}

Therefore, as for the data quality / label noise assessment, I conduct an human evaluation study
for determining the SRL quality: Three people evaluated the SRL quality of examples regarding
whether they estimate them to be helpful, neutral, or harmful to the model for solving
the task at hand.\myfootnote{For single sentence tasks, 20 examples were randomly sampled, for
sentence pair tasks 10 examples, and for question answering tasks 2 examples (since the context
comprises often umpteen sentences.}

In figure \ref{fig:SRL-assessment} the aggregated evaluations per human are summed up:
Evidently, the predicted SRLs were mostly estimated to be neither of great help to the model
nor severely distract it. It has to be noted, however, that the inter-annotator agreement between
the the three people was very low, the computed Fleiss' $\kappa$ was only 0.20 --- indicating
an agreement slightly above randomness.

If the annotations are summed up over each dataset, as in figure \ref{fig:data-set-SRL-assessment},
this impression gets tightened: Only a small subset of the usefulness attributions were unisono,
i.e. all annotators picked the same label (indicated by darker color shades).

\fig{images/SRL_assessment_data_set.pdf}{fig:data-set-SRL-assessment}{Estimated quality of SRLs per dataset.}{15}{SRL assessment per datasets}

Interestingly, for PAWS-X there is comparatively high agreement observable between the annotators:
The bar which indicates harmful judgement is quite high, with a substantial unisono part. However,
as is shown in the right figure of \ref{fig:classification-gains}, PAWS-X is one of the datasets
where there was a clear performance gain from adding SRL information. This is another sign which
indicates that it is difficult for humans to estimate if a certain information will be helpful for
a machine learning model --- even if the intuitive judgement of the quality of the SRLs is rather
negative, there might still be patterns which are of help for an algorithm for solving the task at
hand.



% Fleiss' $\kappa$ = 0.2048 --- this slightly above the threshold of «fair agreement», as defined by \citep{landis1977measurement} (0.20).

% The $\kappa$ for helpful vs. other is even worse: 0.1944

% for individual datasets:

% deISEAR: 0.0814

% SCARE: 0.2401

% PAWS-X: 0.1245

% XNLI: 0.2475

% MLQA: -0.3636

% XQuAD: -0.5



\section{Ablation study}
\label{sec:ablation}

To be able to make substantial claims about the positive influence about a new algorithm over an
established one, it is common ground to conduct an ablation study. In such a study, one tries to
determine which aspects of the proposed architecture contribute how much to the overall performance
gain (or loss, respectively).

\begin{wrapfigure}[17]{r}{0pt}
\begin{minipage}{0.45\linewidth}
  \centering
    \begin{srl}
      \begin{BVerbatim}[commandchars=\\\{\}, fontsize=\footnotesize]
    Ich        \colorbox{llight-blue}{B-A0}         \colorbox{white}{O}
    weiß       \colorbox{blue}{B-V}          \colorbox{white}{O}
    nicht      \colorbox{white}{O}            \colorbox{white}{O}
    ob         \colorbox{llight-blue}{B-A1}         \colorbox{white}{O}
    er         \colorbox{llight-blue}{I-A1}         \colorbox{llight-blue}{B-A0}
    danach     \colorbox{llight-blue}{I-A1}         \colorbox{white}{O}
    in         \colorbox{llight-blue}{I-A1}         \colorbox{llight-blue}{B-A1}
    Augusta    \colorbox{llight-blue}{I-A1}         \colorbox{llight-blue}{I-A1}
    geblieben  \colorbox{llight-blue}{I-A1}         \colorbox{blue}{B-V}
    ist        \colorbox{llight-blue}{I-A1}         \colorbox{white}{O}
    .          \colorbox{white}{O}            \colorbox{white}{O}
    ==============================
    Er         \colorbox{llight-blue}{B-A0}
    wohnte     \colorbox{blue}{B-V}
    weiterhin  \colorbox{white}{O}
    in         \colorbox{llight-blue}{B-A1}
    Augusta    \colorbox{llight-blue}{I-A1}
    .          \colorbox{white}{O}
      \end{BVerbatim}
      \caption{Regular SRLs.}
      \label{srl:normal}
    \end{srl}
\end{minipage}
\end{wrapfigure}

In my case, i.e. the attempt to improve the performance of BERT regarding NLU tasks, the
following question would need some ablation experiments to be answered: What part of the SRLs
is most responsible for the performance boost? To be able to formulate this in a matter which
can be experimentally tested, I identify two easily separable and testable aspects of SRLs,
take for example the sentence depicted in figure \ref{srl:normal}: Firstly, the information
what parts of a sentence are the predicates. The intuition behind this is that maybe the
head relies mostly on the information as to which tokens carry information about the events
that happen in a given sentence. To test this, I drop the information about all SRLs,
except the information that a token is a predicate (left side of figure \ref{srl:ablation}).
In the second case, the hypothesis is reversed: Maybe the head is able to get the most useful
hints from the information which indicates what role certain token groups play in a given
sentence. To test for this all information about predicates is dropped and only information
about arguments is preserved (right side of figure \ref{srl:ablation}).

To see the influence of these SRL ghosting techniques, for each classification dataset, I take a
configuration from table \ref{tab:results} where +SRL showed positive influence and recomputed 5
models for each configuration (preserving only predicate information vs. only argument information).
In table \ref{tab:srl-abla} I report the results from this ablation experiments.


\tab{tab:srl-abla}{Ablation on effect of PREDs and ARGs information isolated. Note that mostly only the combination of both information structures leads to a significant improvement over vanilly BERT embeddings ($-$SRL)}{
  \scalebox{0.9}{
    \begin{tabular}{llcccc}
                     &                               & \multicolumn{1}{c|}{$-$SRL}                  & \multicolumn{3}{c}{+SRL}                                                                                                 \\
                     &                               & \multicolumn{1}{c|}{}                        & \multicolumn{1}{c}{only PREDs}         & \multicolumn{1}{c}{only ARGs}           & \multicolumn{1}{c}{normal}            \\ \cline{3-6}
    deISEAR $\alpha$ & FFNN Head subtok. zeros       & \multicolumn{1}{c|}{\textit{70.86}}          & \multicolumn{1}{c}{72.19}              & \multicolumn{1}{c}{\underline{75.50**}} & \multicolumn{1}{c}{\textbf{77.48**}}  \\
    SCARE $\alpha$   & [CLS] Head merged duplicate   & \multicolumn{1}{c|}{\textit{83.33}}          & \multicolumn{1}{c}{84.47}              & \multicolumn{1}{c}{\underline{85.23}}   & \multicolumn{1}{c}{\textbf{85.61*}}   \\
    PAWS-X $\beta$   & [CLS] Head merged duplicate   & \multicolumn{1}{c|}{\textit{79.92}}          & \multicolumn{1}{c}{80.53}              & \multicolumn{1}{c}{\underline{80.68}}   & \multicolumn{1}{c}{\textbf{82.51***}} \\
    XNLI $\beta$     & GRU Head subtok. zeros        & \multicolumn{1}{c|}{\textit{66.84}}          & \multicolumn{1}{c}{67.02}              & \multicolumn{1}{c}{\textbf{68.00}}      & \multicolumn{1}{c}{\underline{67.82}} \\
    \end{tabular}
  }
}{Ablation Study}


As can be seen, SRL information only displays its full positive effect when both information parts are combined: Providing
the model only with predicate information already leads to a slight, but insignificant improvement in all experiments.
By telling the model exclusively about where arguments are located in sentences leads to a greater advancement, but
mostly also insignificant. For three out of four datasets, informing the model about both SRL information structures
in combination leads to a further significant increase of accuracy.

Apparently, the crucial aspect of information the GliBERT heads improve from having knowledge
about semantic roles lies in its relational character: It is not so much an ``annotation''
of certain parts of sentences --- as for e.g. named entities ---, but the indication of the
relationship in which those sequences stand to each other.

\begingroup
\begin{srl}[!h]
\centering
  \begin{minipage}{0.45\linewidth}
  \vspace{0pt}
    \begin{BVerbatim}[commandchars=\\\{\}, fontsize=\footnotesize]
    Ich        \colorbox{white}{O}           \colorbox{white}{O}
    weiß       \colorbox{blue}{B-V}         \colorbox{white}{O}
    nicht      \colorbox{white}{O}           \colorbox{white}{O}
    ob         \colorbox{white}{O}           \colorbox{white}{O}
    er         \colorbox{white}{O}           \colorbox{white}{O}
    danach     \colorbox{white}{O}           \colorbox{white}{O}
    in         \colorbox{white}{O}           \colorbox{white}{O}
    Augusta    \colorbox{white}{O}           \colorbox{white}{O}
    geblieben  \colorbox{white}{O}           \colorbox{blue}{B-V}
    ist        \colorbox{white}{O}           \colorbox{white}{O}
    .          \colorbox{white}{O}           \colorbox{white}{O}
    ==============================
    Er         \colorbox{white}{O}
    wohnte     \colorbox{blue}{B-V}
    weiterhin  \colorbox{white}{O}
    in         \colorbox{white}{O}
    Augusta    \colorbox{white}{O}
    .          \colorbox{white}{O}
    \end{BVerbatim}
  \end{minipage}
  \hfill
  \begin{minipage}{0.45\linewidth}
  \vspace{0pt}
    \begin{BVerbatim}[commandchars=\\\{\}, fontsize=\footnotesize]
      Ich        \colorbox{llight-blue}{B-A0}         \colorbox{white}{O}
      weiß       \colorbox{white}{O}            \colorbox{white}{O}
      nicht      \colorbox{white}{O}            \colorbox{white}{O}
      ob         \colorbox{llight-blue}{B-A1}         \colorbox{white}{O}
      er         \colorbox{llight-blue}{I-A1}         \colorbox{llight-blue}{B-A0}
      danach     \colorbox{llight-blue}{I-A1}         \colorbox{white}{O}
      in         \colorbox{llight-blue}{I-A1}         \colorbox{llight-blue}{B-A1}
      Augusta    \colorbox{llight-blue}{I-A1}         \colorbox{llight-blue}{I-A1}
      geblieben  \colorbox{llight-blue}{I-A1}         \colorbox{white}{O}
      ist        \colorbox{llight-blue}{I-A1}         \colorbox{white}{O}
      .          \colorbox{white}{O}            \colorbox{white}{O}
      ==============================
      Er         \colorbox{llight-blue}{B-A0}
      wohnte     \colorbox{white}{O}
      weiterhin  \colorbox{white}{O}
      in         \colorbox{llight-blue}{B-A1}
      Augusta    \colorbox{llight-blue}{I-A1}
      .          \colorbox{white}{O}
    \end{BVerbatim}
  \end{minipage}
\end{srl}
\label{srl:ablation}
\captionof{srl}{\textbf{Left}: Only predicate SRLs. \textbf{Right}:  Only argument SRLs.}
\endgroup


\newpage
\section{Summary}

To wrap up my findings, I conclude this chapter by summarizing the afore described observed
phenomena and findings. The structure of the following synopsis adheres to the outlined
research questions from chapter \ref{chap:1_intro}, \ref{sec:research-question}.


\paragraph*{General effects of SRL information on BERT embeddings}

For classification datasets, a modest, but clear gain of 0.8\% in overall accuracy
was determined.\myfootnote{Only the significant results from table \ref{tab:gain-loss}
were taken into account} For question answering datasets no positive, measurable effect
could be asserted.

\cite{zhang2019semantics}, who implemented SRL-enriching for the English BERT model, report an
overall performance increase of SemBERT of 2.4\% on the GLUE classification datasets. Since the
authors do not report significance and the model selection approach that led to these results,
it is difficult to explain the performance gap between the two models. However, as is indicated
by the results of the CoNLL '09 \citep{hajivc2009conll} , English Semantic Role Labelling is
significantly better than for German; which would partly explain the superiority of SemBERT over
GliBERT regarding performance on NLU datasets.

Further, GliBERT showed varying performance according to the task at hand: The most reliable
gains were found on sentence pair classification datasets, followed
by single-sentence classification tasks. For question answering tasks no positive
effect of SRL information could be measured.

However, not only the tasks at hand influences the effectiveness of SRL information,
also the quality of the data has a decisive influence of the performance of the
combined representations: Especially datasets that were created by (semi-)automatically
translate English corpora into a target language expose translation noise, sometimes
leading to wrong gold labels, which hinder reliable model fitting and selection. Therfore,
more data of higher quality would most likely results in constantly higher overall performance.

As the ablation study revealed, a pivotal benefit SRLs offer to plain contextualized embeddings is
the reliable, structural encoding of functional, relational links that hold between the different
agents in a predicatively denoted proposition: The essential semantic functions that are expressed
between agents, predicates, and patients, for example, are captured in the SRLs and support a
subsequent head in solving the down-stream task.


\paragraph*{Differences in head architectures}

The BERT vanilla head for classification adapted to the GLiBERT [CLS] head showed mixed results:
While producing 11 improvements out of 16 experiments, 4 of them significantly, there were also
3 instances where adding SRL information led to a performance decrease, where in two thirds of
the cases, this decrease was statistically significant. The [CLS] head therefore is a relatively
stable head, however, if the random initialization of the weights is not optimal or it gets stuck
in a non-optimal minimum of the objective function, it produces severe negative results.

The FFNN head, where all token embeddings get concatenated is clearly the worst classification
head: The positive and negative effects of adding SRL information almost balance each other out
--- there is a slight bias for positive results.

Processing the token embeddings by a recurrent architecture is the most reliable approach: The GRU
head improves in 75\% of all experiments when SRL information is added. Often these gains are not
statistically significant when considered in isolation, but the overall tendency of the GRU head
clearly suggests its superior capability of processing the NLU supporting character of SRLs.

For question answering, the GliBERT adapted vanilla head does not benefit from embedded SRL
information --- all improvements were statistically insignificant and in the majority of the
experiments, adding SRL information led to a performance decrease.


\paragraph*{Differences of SRL implementation modes}

When a sentence has less than three predicate-argument structures, there were two
approaches presented as well: Duplicating the first predicate-argument structure to
fill the empty slots versus filling the empty predicate-argument structures with a
special \texttt{0}-SRL. Here, duplicating the first predicate-argument structure has
shown to be slightly more effective than the zero-filling approach on most datasets:
Only for XNLI and the question answering datasets the duplicating strategy showed a
weaker performance in tendency compared to the zero-filling approach.

For merging the SRL embeddings with the BERT embeddings, two modes were proposed: Merging
BERT subtokens back to token level by averaging the respective vectors, or duplicating the
SRLs to match up with the BERT subtokens. Over all datasets, and for question answering
datasets in particular, duplicating the SRLs and preserving the BERT subtoken embeddings
clearly led to better results than the latter approach. Presumably, this has to do with
the inevitable information loss during the subtokens avering process.


