\documentclass[a4paper, 12pt,numbers=noenddot]{scrreprt}

\usepackage[utf8]{inputenc}
\usepackage[LGR,T1]{fontenc}
\usepackage[scaled]{helvet}
\usepackage[ngerman,english]{babel} %Trennungen, Schriftsatz; Neue deutsche Rechtschreibung
\usepackage[square]{natbib}  % more options for citations, e.g. here: square brackets as ``parentheses''
\renewcommand{\bibsection}{}

\usepackage[pdftex]{graphicx}

\usepackage{qtree}
\usepackage{wrapfig}
\usepackage{multirow}
\usepackage[table]{xcolor}

\usepackage{booktabs} % mehr Optionen zur Gestaltung von Tabellen
\usepackage{covington} % 'examples' Umgebung fuer linguistische Beispiele

% own packages %%%%%%%%%%%%%%%%%%%%%%%%%%%

% \usepackage{polyglossia}
% \setmainlanguage{english}
% \setotherlanguages{sanskrit}
% \newfontfamily\devanagarifont[Script=Devanagari]{Lohit Devanagari}
% \usepackage{devanagari}

\usepackage{algorithm}
\usepackage{algorithmic}
\usepackage{amsmath}
\usepackage{amssymb}
\usepackage{bbm}
\usepackage{csquotes}
\usepackage{enumitem}
\usepackage{epigraph}
\usepackage{fancyvrb}
\usepackage{float}
\usepackage[font=footnotesize,labelfont=bf]{caption}
\usepackage{footnote}
\usepackage{hhline}
\usepackage{makecell}
\usepackage{multicol}
\usepackage{pdflscape}
\usepackage{siunitx}
\usepackage{tabularx}
\makesavenoteenv{tabular}
\makesavenoteenv{table}
\usepackage[most]{tcolorbox}
\usepackage{textcomp}
\usepackage{tikz}
\usepackage{tipa}
\usepackage{titlesec}
\usepackage{wrapfig}
\usepackage{xcolor}

\definecolor{light-gray}{gray}{0.70}
\definecolor{llight-gray}{gray}{0.90}
\definecolor{light-blue}{HTML}{C6DBEF}
\definecolor{llight-blue}{RGB}{222, 235, 247}
\definecolor{blue}{HTML}{9ECAE1}
\definecolor{dark-blue}{HTML}{4292C6}
\definecolor{std-blue}{HTML}{4169E1}

\definecolor{yellow}{HTML}{FED976}
\definecolor{manatee}{HTML}{9F9FAD}
\definecolor{sage}{HTML}{B8B08D}
\definecolor{purple}{HTML}{BCBDDC}
\definecolor{green}{HTML}{A1D99B}
\definecolor{std-green}{HTML}{006400}
\definecolor{red}{HTML}{FC9272}
\definecolor{std-red}{HTML}{FF0000}

% own commands

\renewcommand\labelitemi{--}
\addtolength{\skip\footins}{1pc plus 1pt}

\newcolumntype{g}{>{\columncolor{llight-gray}}c}
\newcolumntype{P}[1]{>{\centering\arraybackslash}p{#1}}

\newcommand{\customcolorbox}[2]{\tikz[baseline=(char.base)]\node[minimum width=2em,text height=1.7ex, text depth=0.1ex,fill=#2,text=black, align=left](char){#1};}

\newcommand{\textgreek}[1]{\begingroup\fontencoding{LGR}\selectfont#1\endgroup}

\floatstyle{plain}
\newfloat{srl}{H}{loe}[chapter]
\floatname{srl}{SRL}

\long\def\wrapfiguresafe#1#2#3{%
  \sbox\curwrapfig{#3}%
  \par\penalty-100%
  \begingroup % preserve \dimen@
    \dimen@\pagegoal \advance\dimen@-\pagetotal % space left
    \advance\dimen@-\baselineskip % allow an extra line
    \ifdim \ht\curwrapfig>\dimen@ % not enough space left
      \break%
    \fi%
  \endgroup%
  \begin{wrapfigure}{#1}{#2}%
    \usebox\curwrapfig%
  \end{wrapfigure}%
}

% own formatting

% \setsansfont{Calibri}
\titleformat{\paragraph}
  {\sffamily}{\thesection}{1em}{}


%%%%%%%%%%%%%%%%%%%%%%%%%%%%%%%%%%%%%%%%%%%

\def\signed #1 (#2){{\unskip\nobreak\hfil\penalty50
  \hskip2em\hbox{}\nobreak\hfil\sl#1\/ \rm(#2)
  \parfillskip=0pt \finalhyphendemerits=0 \par}}

% define toc formatting
\usepackage[titles]{tocloft}
\setlength{\cftsubsecindent}{3em}
\setlength{\cftsubsecnumwidth}{3.3em}
\setlength{\cftsubsubsecindent}{4.5em}
\setlength{\cftsubsubsecnumwidth}{4em}



% figure numbering
\newcounter{myfigure}
\renewcommand{\thefigure}{\arabic{myfigure}}
\newcounter{mytable}
\renewcommand{\thetable}{\arabic{mytable}}
\usepackage{makeidx} \makeindex
\makeglossary


% Formatierung von Referenzen
\usepackage[hyperfootnotes=false]{hyperref}
\hypersetup{%
  pdfauthor={Jonathan Schaber},
  pdftitle={Enriching BERT embeddings with Semantic Role Labels for Natural Language Understanding Tasks in German},
  pdfsubject={},
  pdfkeywords={},
  pdfborder=000
}
% so werden Hyperlinks nicht gefärbt


% Definition vom Seitenlayout
\setlength{\topmargin}{-1.2cm}
\setlength{\oddsidemargin}{0.5cm}
\setlength{\evensidemargin}{0.5cm}

\setlength{\textheight}{24.5cm}
\setlength{\textwidth}{15cm}

\setlength{\footskip}{1.2cm}
\setlength{\footnotesep}{0.4cm}

% Definition vom Header und Footer im Seitenlayout
\usepackage{fancyhdr}
\pagestyle{fancy}
\fancyhf{}

% Linien nach dem Header und vor dem Footer
%\renewcommand{\footrulewidth}{0.4pt}
\renewcommand{\headrulewidth}{0.4pt}

\fancyhead[L]{\footnotesize{\leftmark}}
\fancyhead[R]{}
\fancyfoot[C]{\footnotesize{\thepage}}

% Plain Pagestyle für Kapitelanfangsseite
\fancypagestyle{plain}{%
  \fancyhf{}
  \renewcommand{\headrulewidth}{0pt}
  \fancyfoot[C]{\footnotesize{\thepage}}
}


% Neues Kapitel Makro, damit die Variablen korrekt abgefüllt werden
\newcommand{\newchap}[1]{
	\chapter{#1}
	\markboth {Chapter \thechapter.  {#1}}{Chapter \thechapter.  {#1}}
}

% Footnote numbering
\newcounter{myfootnote}[chapter]
\renewcommand{\thefootnote}{\themyfootnote}
\newcommand{\myfootnote}[1]{\stepcounter{myfootnote}\footnote{\setstretch{1.3}#1}}

% command to import a figure
\newcommand{\fig}[5]{
  \begin{figure}[h]
    \begin{center}
      \includegraphics[width=#4cm]{#1}
    \end{center}
    \stepcounter{myfigure}
    \caption[#5]{#3}
    \label{#2}
  \end{figure}
}

\newcommand{\figtable}[4]{
  \begin{figure}[h]
    \begin{center}
      {
	\footnotesize
	\sffamily
	#3
      }
    \end{center}
    \stepcounter{myfigure}
    \caption[#4]{#2}
    \label{#1}
  \end{figure}
}

\newcommand{\tab}[4]{
  \begin{table}[h]
    \begin{center}
      {
	\footnotesize
	% \sffamily
	\renewcommand{\arraystretch}{1.4}
	#3
      }
    \end{center}
    \stepcounter{mytable}
    \caption[#4]{#2}
    \label{#1}
  \end{table}
}

% command to refere to a figure
\newcommand{\reffig}[1]{Figure \ref{#1}}

% examples as floating environment
\usepackage{float}
\newfloat{example}{tbp}{loe}[chapter]
\floatname{example}{Example}

%\floatplacement{figure}{htbp}
%\floatplacement{table}{htbp}
\floatplacement{fexample}{htbp}

% Definition des Nummerierungslevel
\setcounter{secnumdepth}{4}
\setcounter{tocdepth}{4}
\setcounter{lofdepth}{1}

% Definition von Paragraphen
\parskip=0.3cm
\parindent=0cm

% Definition vom Zeilenabstand
\usepackage{setspace} % Zeilenabstand
\onehalfspacing %\doublespace or \singlespace

% Befehle für Anführungszeichen
\usepackage{xspace} % Leerschlag nach Anführungszeichen
\newcommand{\qr}{\grqq\xspace}
\newcommand{\qrs}{\grqq\ }
\newcommand{\ql}{\glqq}

% Formatierungsbefehle zum Zitieren
\newcommand{\page}[1]{p.~#1}
\newcommand{\pagef}[1]{p.~#1f.}
\newcommand{\pageff}[1]{p.~#1ff.}
\newcommand{\pages}[2]{p.~#1--#2}

\newcommand{\etiq}[1]{\textbf{#1}}
\newcommand{\token}[1]{\textit{#1}}

\usepackage{supertabular}
